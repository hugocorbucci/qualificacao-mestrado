% Arquivo LaTeX de exemplo de dissertação/tese a ser apresentados à CPG
% do IME-USP
% 
% Obs: Leia previamente o texto do arquivo Readme.txt
%

\documentclass[12pt,twoside,letterpaper]{book}

% ---------------------------------------------------------------------------- %
% Pacotes 
\usepackage[brazil]{babel}
\usepackage[utf8]{inputenc}
\usepackage[pdftex]{graphicx}           % usamos arquivos pdf/png como figuras
\usepackage{color}
\usepackage{pifont}
\usepackage{amsfonts}
\usepackage{amssymb} 
\usepackage{setspace}                   % espaçamento flexível
\usepackage[small,compact]{titlesec} 	% cabeçalhos dos títulos: menores e compactos
\usepackage{indentfirst} 				% indentação do primeiro parágrafo
\usepackage{cite}  						% modo de citação
\usepackage{subfigure} 					% uso de várias figuras numa só
\usepackage{makeidx} 					% índice remissivo
\usepackage[nottoc]{tocbibind} 			% acrescentamos a bibliografia/indice/conteudo no Table of Contents
\usepackage{setspace}
\usepackage{longtable}
\usepackage{lscape}
\usepackage{caption}
\usepackage[pagebackref,colorlinks=true,urlcolor=red,citecolor=green,linkcolor=blue]{hyperref}
\usepackage[letterpaper,top=2.54cm, bottom=2.54cm, left=2.54cm, right=2.54cm]{geometry}
\usepackage[toc,page]{appendix}

% ---------------------------------------------------------------------------- %
% Alguns comandos
\graphicspath{{./figuras/}} 			% path das figuras (recomendável)
\makeindex  							% para o índice remissivo
\raggedbottom   						% para não permitir espaços extra no texto
\listfiles  							% lista os arquivos utilizados no latex durante a compilação
\normalsize

% Fontes menores nos captions (para figuras e tabelas)
\newcommand{\captionfonts}{\small}
\makeatletter  % Allow the use of @ in command names
\long\def\@makecaption#1#2{%
  \vskip\abovecaptionskip
  \sbox\@tempboxa{{\captionfonts #1: #2}}%
  \ifdim \wd\@tempboxa >\hsize
    {\captionfonts #1: #2\par}
  \else
    \hbox to\hsize{\hfil\box\@tempboxa\hfil}%
  \fi
  \vskip\belowcaptionskip}
\makeatother   % Cancel the effect of \makeatletter

% para melhorar o posicionamento das figuras no texto
\renewcommand{\topfraction}{0.85}
\renewcommand{\textfraction}{0.1}
\renewcommand{\floatpagefraction}{0.75}

% ---------------------------------------------------------------------------- %
% Corpo do texto
\begin{document}
\frontmatter \onehalfspacing  % espaçamento

% ---------------------------------------------------------------------------- %
% Capa
\thispagestyle{empty}
\begin{center}
    \vspace*{2.3cm}
    \textbf{\Large{Métodos ágeis e software livre:\\
        Dois mundos que deveriam ter uma interação mais intensa}}\\
	
    \vspace*{1.2cm}
    \Large{Hugo Corbucci}
    
    \vskip 2cm
	\textsc{
	Qualificação apresentada\\[-0.25cm] 
	ao\\[-0.25cm]
	Instituto de Matemática e Estatística\\[-0.25cm]
	da\\[-0.25cm]
	Universidade de São Paulo}
    
    \vskip 1.5cm
    Programa: Mestrado em Ciências da Computação\\
    Orientador: Prof. Dr. Alfredo Goldman

    \vskip 1cm
	\normalsize{Durante o desenvolvimento deste trabalho o autor recebeu auxílio
	financeiro do projeto Qualipso}
	
    \vskip 0.5cm
    \normalsize{São Paulo, janeiro de 2009}
\end{center}

% ---------------------------------------------------------------------------- %
% Página de rosto (só para a versão final)
%\newpage
%\thispagestyle{empty}
%	\begin{center}
%    	\vspace*{2.3 cm}
%    	\textbf{\Large{Título do trabalho a ser apresentado à \\
%		CPG para a dissertação/tese}}\\
%	    \vspace*{2 cm}
%	\end{center}
%
%	\vskip 2cm
%
%	\begin{flushright}
%	Este exemplar corresponde à redação\\
%	final da dissertação devidamente corrigida\\
%	e defendida por Hugo Corbucci\\
%	e aprovada pela Comissão Julgadora.
%	\vskip 2cm
%
%	\end{flushright}
%	\vskip 4.2cm
%
%	\begin{quote}
%	\noindent Banca Examinadora:
%	
%	\begin{itemize}
%		\item Prof. Dr. Alfredo Goldman (orientador) - IME-USP.
%		\item Prof. Dr. Fabio Kon - IME-USP.
%		\item Prof. Dr. Nome Completo - IMPA.
%	\end{itemize}
%	  
%	\end{quote}
%\pagebreak

\pagenumbering{roman} 	% começamos a numerar 

% ---------------------------------------------------------------------------- %
% Agradecimentos
\chapter*{Agradecimentos}

Este trabalho contou com o apoio do projeto Qualipso \cite{Qualipso}.

Gostaria de agradecer ao Christian Reis por sua ajuda, pelas
discussões interessantes e pelo apoio.

% ---------------------------------------------------------------------------- %
% Resumo
\chapter*{Resumo}

Métodos ágeis e comunidades de software livre compartilham culturas
similares mas com diferentes abordagens para superar
dificuldades. Apesar dos diversos profissionais envolvidos em ambos
mundos, os métodos ágeis não são tão fortes quanto poderiam na
comunidade de software livre nem o contrário. Esse trabalho identifica
e expõe os obstáculos que separam essas comunidades para extrair as
melhores soluções de cada uma e contribuir com sugestões de
ferramentas e processos de desenvolvimento em ambas comunidades.

\noindent \textbf{Palavras-chave:} métodos ágeis, open source,
software livre

% ---------------------------------------------------------------------------- %
% Abstract
\chapter*{Abstract}

Agile methods and open source software communities share similar
cultures with different approaches to overcome problems. Although
several professionals are involved in both worlds, neither agile
methodologies are as strong as they could be in open source
communities nor those communities provide strong contributions to
agile methods. This work identifies and exposes the obstacles that
separate those communities in order to extract the best of them and
improve both sides with suggestions of tools and development
processes.

\noindent \textbf{Keywords:} agile methods, open source

% ---------------------------------------------------------------------------- %
% Sumário
\tableofcontents % imprime o sumário

% ---------------------------------------------------------------------------- %
% Listas: abreviaturas, símbolos, figuras e tabelas

\chapter{Lista de Abreviaturas}
\begin{tabular}{ll}
 		SL       & Software Livre.\\
 		OSS         & Software de código aberto (\emph{Open Source Software}).\\
 		XP       & Programação Extrema (\emph{Extreme Programming}).\\
 		FLOSS       & Software Gratuito, Livre e de código aberto (\emph{Free, Libre and Open Source
 Software}).\\
 		BDD       & Desenvolvimento Dirigido por Comportamento (\emph{Behaviour Driven Development}).\\
 		IRC       & Papo Retransmitido pela Internet (\emph{Internet Relay Chat}).\\
 		FISL       & Fórum Internacional de Software Livre.\\
 		API       & Interface de Programação da Aplicação (\emph{Application Programming Interface}).\\
\end{tabular}

%\chapter{Lista de Símbolos}
%\begin{tabular}{ll}
%		$\omega$    & Freqüência angular.\\
%\end{tabular}

%\listoffigures               % lista de Figuras
%\listoftables                % lista de Tabelas

% ---------------------------------------------------------------------------- %
% Capítulos
\mainmatter

%\singlespacing              % espaçamento simples
\onehalfspacing              % espaçamento um e meio
%\doublespacing              % espaçamento duplo

\input cap-introducao        % associado ao arquivo: 'cap-introducao.tex'
\input cap-definicoes        % associado ao arquivo: 'cap-definicoes.tex'
\input cap-open-to-agile     % associado ao arquivo: 'cap-open-to-agile.tex'
\input cap-agile-to-open     % associado ao arquivo: 'cap-agile-to-open.tex'
\input cap-perspectivas     % associado ao arquivo: 'cap-perspectivas.tex'
\input cap-conclusoes        % associado ao arquivo: 'cap-conclusoes.tex'
\appendix
\input ape-pesquisaEA      % associado ao arquivo: 'ape-pesquisaEA.tex'
\input ape-pesquisaOS      % associado ao arquivo: 'ape-pesquisaOS.tex'
\input ape-pesquisaMA      % associado ao arquivo: 'ape-pesquisaMA.tex'

% ---------------------------------------------------------------------------- %
% Bibliografia
\backmatter \singlespacing   % espaçamento simples

\bibliographystyle{amsplain}
\bibliography{bibliografia}  % associado ao arquivo: 'bibliografia.bib'

% ---------------------------------------------------------------------------- %
% Índice remissivo
%\index{TBP|see{periodicidade região codificante}}

\printindex   % imprime o índice remissivo no documento 

\end{document}

