%% ------------------------------------------------------------------------- %%
\chapter{Métodos Ágeis no Contexto do Software Livre}
\label{cap:agile}

Na Agile 2008, Mary Poppendieck conduziu um
\emph{workshop}\footnote{http://submissions.agile2008.org/node/376 -
  Acessado em 16/03/2009} com Christian Reis intitulado \emph{``Open
  Source Meets Agile - What can each teach the other?''}. Seu objetivo
era de discutir práticas de sucesso em um projeto de software livre
que não eram encontradas em métodos ágeis. Desta forma, os
participantes poderiam compreender alguns princípios essenciais que se
aplicam a projetos de software livre e poderiam propor melhorias aos
atuais métodos ágeis. Um pequeno resumo da discussão pode ser
encontrado na seção \ref{sec:foss-over-agile}.

De um outro ponto de vista, faltam soluções especiais para o
desenvolvimento de software livre nos métodos ágeis mais conhecidos
atualmente. A Seção \ref{sec:agile-improve-os} apresenta como a
criação dessa solução poderia ajudar tanto o software livre quanto a
comunidade de métodos ágeis.

\section{Princípios do Software Livre Interesantes em Métodos Ágeis}
\label{sec:foss-over-agile}

Reis é um desenvolvedor Brasileiro de software livre que trabalha para
a Canonical Inc. no desenvolvimento do LaunchPad
\footnote{http://launchpad.net/ - Last accessed 24/04/2009}, o projeto
de gerenciamento de software para a distribuição Linux Ubuntu. O
\emph{workshop} teve início com a apresentação de Reis sobre como o
LaunchPad é desenvolvido. Três pontos essenciais foram levantados
durante a discussão que deu seqüência à apresentação e será descrita
na próxima subseção. O primeiro (Subseção \ref{subsec:commiter})
descreve e discute o papel de \emph{commiter}.  O segundo (Subseção
\ref{subsec:publicity}) apresenta os benefícios de seguir um processo
de desenvolvimento que seja público e transparente.  Por fim, o último
(Subseção \ref{subsec:crossrev}) aborda o sistema de revisão cruzada
dos sistemas que é usado para garantir a comunicação e a clareza do
código.

\subsection{O Papel do \emph{Commiter}}
\label{subsec:commiter}

Parte do valor que foi identificado no software livre foi o papel do
\emph{commiter}. Como esse papel tem uma relação relativamente
complicada com métodos ágeis, essa subseção será divida em quatro
partes. A primeira descreve o que é um \emph{commiter}, a segunda
apresenta como esse papel é distribuído em métodos ágeis, a terceira
aborda as diferenças e semelhanças entre a revisão realizada durante a
programação em pares e a revisão feita pelo \emph{commiter} e, por
fim, a quarta apresenta as sugestões de adaptação desse papel em
métodos ágeis.

% TODO Ajustar o conteúdo de cada uma das subsubseções.

\subsubsection{O que é um \emph{commiter}}

Um \emph{commiter} é uma pessoa que tem direito de adicionar,
modificar e remover código fonte ao ``galho''\footnote{Um galho
  (\emph{branch}) de um repositório é uma ramificação da estrutura de
  diretórios que guarda os arquivos} principal do repositório de
controle de versões. O ``galho'' principal é a parte do código que
será empacotada para formar uma nova versão do programa. Aos olhos da
comunidade do software, o \emph{commiter} é uma pessoa confiável muito
qualificada para avaliar a qualidade do código fonte. Este é o meio
encontrado pelas comunidades de software livre para revisar a grande
maioria do código fonte de forma a reduzir a quantidade de erros e
melhorar a clareza do código.

A maioria dos projetos de software livre tem um grupo muito pequeno de
\emph{commiters}. Frequentemente o líder do projeto é o único
\emph{commiter} e todos os \emph{patches} devem passar por sua
aprovação. De acordo com Riehle \cite{Riehle2007}, existem três níveis
na hierarquia tradicional de um projeto de software livre.
\begin{itemize}
\item O primeiro nível é o de usuário.

  Usuários têm o direito de usar o programa, relatar problemas e pedir
  funcionalidades.
\item O segundo nível é o de contribuidor.

  A promoção entre o primeiro e o segundo nível é implícita. Ela
  acontece quando um \emph{commiter} aceita os \emph{patches} do
  usuário e os envia ao repositório de código no ``galho''
  principal. Normalmente, ninguém, exceto o \emph{commiter} e o
  contribuidor, sabe dessa promoção.
\item O terceiro papel é o de \emph{commiter}.

  Neste nível, a transição é explícita. Contribuidores e
  \emph{commiters} demonstram apoio a uma determinada pessoa e
  reconhecem publicamente a qualidade geral de seu trabalho. Por isso,
  atingir o nível de \emph{commiter} é um feito valioso que significa
  que essa pessoa produz código de ótima qualidade e está realmente
  envolvido com o desenvolvimento do projeto.
\end{itemize}

\subsubsection{O papel do \emph{commiter} em métodos ágeis}

Métodos ágeis delegam o papel do \emph{commiter} para cada um dos
desenvolvedores da equipe. No \emph{workshop} sugeriram que alguma
forma de controle no ``galho'' principal de um projeto ágil poderia
melhorar ainda mais a simplicidade do código fonte do aplicativo de
produção.

Na maioria dos métodos ágeis, uma equipe deveria ter um líder (um
\emph{Scrum Master} em Scrum, um treinador em XP, etc...)  que é mais
experiente naquele método ágil que o resto da equipe. O líder da
equipe é responsável por lembrar a equipe de se ater às práticas
escolhidas. Ele também deve ajudar a equipe a resolver os problemas
encontrados e idealmente, transformar todos os membros da equipe em
possíveis líderes de forma a tornar-se ``inútil''.

Para cumprir essa função, o líder não precisa obrigatoriamente ter
conhecimentos técnicos apurados. No entanto, uma equipe de
desenvolvimento costumeiramente precisa de ajuda do ponto de vista
técnico em alguma parte de seu trabalho. Alguns dos problemas
levantados por uma equipe podem ser causados por decisões ou por
dificuldades técnicas. Neste caso, se o líder não tiver conhecimento
técnico, ele pode encontrar dificuldades para cumprir sua função. Para
resolver este problema, é comum que o líder tenha a ajuda de um
consultor técnico que pode ser um membro da equipe ou uma pessoa de
fora.

Se este consultor técnico for um membro da equipe, ele tem,
indiretamente, a responsabilidade de fazer com que a equipe mantenha
uma boa qualidade de código. Pensando assim, o responsável técnico tem
a função de \emph{commiter} do projeto mas realiza seu trabalho
lembrando aos programadores de que seu código deve estar sempre
legível, claro e com testes passando.

\subsubsection{Semelhanças e diferenças da revisão}

O papel ativo de revisor que o \emph{commiter} tem em projetos de
software livre é encontrado no co-piloto de uma dupla de programação
em pares. Note, no entanto, que a revisão de código realizada durante
a programação em pares tem como objetivo principal a redução de erros
e não é obrigatoriamente eficiente no aumento da clareza do
código. Isso se dá porque, quando um par trabalha em uma tarefa, ambas
pessoas mergulham em um determinado trecho de código e criam juntas
uma linha de pensamento. Para ambos os envolvidos, o tal trecho código
pode ser muito claro graça ao contexto e à linha de pensamento que
eles criaram. Mas, para alguém que não acompanhou essa linha, o código
pode ser muito complexo se ele não deixar indício do pensamento que
deve ser seguido.

A revisão feita pelo \emph{commiter} dificilmente será mais eficiente
que a do par para reduzir a quantidade de erros já que o revisor
costuma ter menos tempo para pensar sobre o problema e entender os
possíveis casos envolvidos. Enquanto o par que trabalhou no código
teve exatamente este objetivo. No entanto, o \emph{commiter} traz um
olhar fresco ao código que é muito mais semelhante ao olhar de um
desenvolvedor qualquer no futuro. Deste ponto de vista, é mais
provável que o revisor questione o código de forma semelhante àquela
que outra pessoa no futuro faria. Sendo assim, o \emph{commiter} pode
evitar os principais problemas relacionados à clareza do código
produzido.

De qualquer forma, o trabalho de revisão tem duas consequências
diretas e evidentes. A primeira é de que o tempo necessário para que
uma mudança seja incorporada ao ``galho'' principal do código aumenta
consideravelmente já que, tipicamente, são necessárias algumas
conversas entre o revisor e os autores do código. A segunda é que o
trabalho do revisor, se ele for único, é considerável já que ele deve
ler todo código que deve ir para o ``galho'' principal, tentar
entendê-lo e expressar suas dúvidas aos autores.

\subsubsection{Sugestões para adaptar o papel aos métodos ágeis}

Considerando os pontos apresentados no fim da seção anterior, dar o
papel de \emph{commiter} ao consultor técnico de uma equipe ágil
significaria criar um gargalo de incorporação de código. A Teoria das
Restrições %TODO Achar referência
(que é muito ligada aos métodos ágeis) afirma que deve-se eliminar os
gargalos para maximizar a produtividade de um equipe.

Sendo assim, a proposta é de manter um pequeno conjunto de
desenvolvedores da equipe como \emph{commiters} e fazer o papel
circular entre os membros da equipe. Ao trocar os membros do conjunto
de \emph{commiters}, permite-se uma maior distribuição do conhecimento
e reduz-se a aparente concentração de poder desse papel. A troca
também permite que aqueles que foram \emph{commiters} possam, por sua
vez, serem autores de alguns trechos de código que passarão por
avaliação de outros. Desta forma, toda a equipe passa a entender o
valor de cada um dos papéis e entende melhor como escrever código que
seja claro para um revisor.

\subsection{Resultados Públicos}
\label{subsec:publicity}

Outro ponto importante da discussão foi a divulgação pública de todos
os resultados relacionados ao projeto. De acordo com Reis, programas
proprietários também podem se beneficiar de um sistema de rastreamento
de erros público e da publicação dos resultados dos testes
automatizados. A contra-parte para abraçar os benefícios dessas
práticas é o de expor alguns detalhes de código. Disponibilizar esses
resultados publicamente encoraja os usuários a participar do processo
de desenvolvimento já que eles entendem como e quando o programa é
melhorado.

Em métodos ágeis, o resultado dos testes e a lista fornecida pelo
sistema de rastreamento de erros são informações muito importantes
para a equipe de desenvolvimento. Apesar disso, nenhum métodos afirma
explicitamente que o cliente e os usuários deveriam estar em contato
direto com essas ferramentas.

É senso comum em métodos ágeis que o cliente deveria ser parte da
equipe de desenvolvimento. Como a equipe deve estar sempre em contato
com essas ferramentas, pode-se interpretar que o cliente deveria usar
a ferramenta de forma semelhante ao resto da equipe. Infelizmente, a
maioria das ferramentas usadas são muito rudimentares do ponto de
vista de um cliente não-técnico já que poucas delas se preocupam em
atribuir um significado de negócios aos resultados.

Algumas iniciativas\footnote{RSpec - http://rspec.info/ - Último
  acesso em 30/09/2008}$^{, }$\footnote{JBehave - http://jbehave.org/
  - Último acesso em 30/09/2008} relacionadas aos testes já existem
ligadas ao movimento de Desenvolvimento Dirigido pelo Comportamento
(\emph{BDD - Behaviour Driven Development}) \cite{North2006} para
produzir melhores relatórios. Já no ponto de vista dos sistema de
rastreamento de erros, a evolução não aconteceu pontualmente mas as
ferramentas mais recentes tendem a apresentar uma interface com menos
detalhes técnicos para alguns usuários (clientes).

Mas a divulgação pública de informações relacionadas ao projeto não se
restringe aos erros ou aos testes. Nas comunidades de software livre,
as discussões entre os membros do projeto e até as discussões com
pessoas de fora do projeto sempre são guardadas no histórico da lista
de correio eletrônico usada. Discussões fora dessa lista são
fortemente desencorajadas já que elas impedem outras pessoas de
contribuir com comentários e idéias. Os históricos das listas ajudam a
construir uma documentação para futuros usuários assim como criar um
rápido sistema de \emph{feedback} para novatos.

Além disso, manter o histórico da lista também inibe atitudes
desrespeituosas já que todas as discussões são salvas e guardadas para
acesso futuro. Desta forma, os participantes costumam manter o
respeito (que é importantíssimo para o sucesso de qualquer projeto)
entre eles e com novatos.

A rastreabilidade é um dos pontos fracos dos métodos ágeis. A maioria
dos métodos sugere que o projeto do software (\emph{design}) evolua
com o tempo conforme as necessidades. Essa evolução deveria fluir
naturalmente dos quadros brancos ou \emph{flip charts}. O problema com
essa abordagem é que quadro brancos são apagados e \emph{flip charts}
são reciclados.  Mesmo quando estes são guardados de alguma forma
(fotos, transcrições ou até mesmo no código), as discussões que
levaram à solução são perdidas.

A fala é uma forma muito eficiente de comunicação mas também muito
efêmera. Mesmo quando uma conversa é gravada, é difícil buscar
informações sobre algum trecho da discussão. Correios eletrônicos são
muito menos eficientes para a comunicação mas têm um grande ganho na
facilidade de busca. Num curto prazo, é evidente que a conversa é
muito mais eficiente para transmitir idéias que a escrita,
especialmente em equipes pequenas. No entanto, num médio ou longo
prazo, os ganhos da comunicação escrita podem superar (como eles o
fazem em projetos livres) as perdas.

\subsection{Revisão Cruzada}
\label{subsec:crossrev}

O terceiro ponto que Reis apresentou foi bem específico ao
LaunchPad. Como o LaunchPad é uma plataforma usada por outras equipes
para que elas desenvolvam seus próprios projetos, quando há uma
mudança na Interface de Programação da Aplicação (\emph{API -
  Application Programming Interface}), um membro de uma equipe externa
que usa o programa (preferenciamente uma pessoa diferente a cada vez)
deve revisar a mudança da interface e os motivos que levaram a
ela. Essa mudança não pode ser enviada ao ``galho'' principal do
repositório a não ser que o revisor externo a aprove. Essa prática é
conhecida como revisão cruzada das mudanças de API ou, simplesmente,
uma revisão cruzada.

Essa prática mata alguns coelhos em uma só cajadada. O papel do
\emph{commiter} resolve o problema da revisão de código que os métodos
ágeis atacam com a programação em pares. A revisão cruzada garante que
a mudança da interface é aprovada pelos usuários assim como os
desenvolvedores.

Ela também garante uma melhora considerável sobre aquela API já que a
conversa entre o desenvolvedor do projeto e o usuário é arquivada pela
lista de correio eletrônico. Desta forma, futuros usuários ou mesmo
outros usuários atuais podem ler e entender porque a API mudou e como
usá-la quando for necessário. Também fica mais fácil realizar mudanças
no futuro e simplificações já que fica claro o que aquela API está
querendo permitir e se aquilo ainda faz sentido nas novas versões.

Por fim, a revisão cruzada também ajuda a envolver o cliente nas
decisões de arquitetura da solução e garante que ele está de acordo
com as mudança realizada. Com isso, é mais fácil identificar um
possível problema de requisitos e corrigí-lo antes que eles sejam
implementados na base principal de código. Obviamente, esta prática só
pode se aplicar até um certo nível quando o usuário não tem
conhecimento técnico. Uma revisão externa pode ajudar a garantir a
clareza da API e a documentar as mudanças mas ela não vai identificar
problemas de requisitos se o revisor não for um cliente ou usuário.

\section{Contribuições de Métodos Ágeis no Software Livre}
\label{sec:agile-improve-os}

A maioria dos problemas apontados até agora são relacionados a
dificuldades de comunicação causados pela quantidade de pessoas
envolvidas no projeto separação física e seus vários conhecimentos e
culturas. Apesar desses fatores serem levados ao extremo em projetos
de software livre, equipes de métodos ágeis distribuídas encontram
alguns dos mesmos problemas \cite{Sutherland2007,Maurer2002}.

Como Beck sugere \cite{Beck2008}, ferramentas pode melhorar a adoção e
o uso de práticas ágeis e, desta forma, melhorar o processo de
desenvolvimento. Uma quantidade considerável de trabalho já foi
realizado na questão de ferramentas da programação em pares
distribuída\footnote{http://sf.net/projects/xpairtise/ - Last accessed
  02/10/2008} e estudos a respeito \cite{Nagappan2003} mas pouco tem
sido produzido para apoiar outras práticas. Como o problema está
relacionado à comunicação, algumas práticas de métodos ágeis são
relevantes. As próximas subseções vão apresentar essas práticas e as
ferramentas sugeridas para facilitar a adoção de métodos ágeis na
comunidade de software livre.

\subsection{Ambiente Informativo}
\label{subsec:inform-worksp}

Essa prática sugere que uma equipe de métodos ágeis deveria trabalhar
num ambiente que provê informações relacionadas ao trabalho. Beck
\cite{XP01} atribui um papel específico, o de acompanhador, para uma
pessoa (ou algumas pessoas) que deve manter essa informação disponível
e atualizada para a equipe. Com equipes co-locadas, o acompanhador
normalmente coleta métricas \cite{Sato2007} automaticamente e
seleciona algumas delas para apresentá-las no ambiente. A maioria das
métricas objetivas são relacionadas ao código fonte enquanto as
métricas subjetivas costumam depender da opinião dos membros da
equipe.

A coleta destes dados não é uma tarefa árdua mas normalmente consome
um tempo considerável e não agrega um benefício imediato ao projeto. É
provalvemente esse o motivo para a falta de métricas ou dados
atualizados em páginas de projeto de software livre. Uma ferramenta
que poderia melhorar esse cenários seria um sistema baseado em
\emph{plug ins} com um conjunto inicial de métricas e uma forma de
criar e apresentar novas métricas. Essas ferramentas deveriam estar
disponíveis em incubadoras de software livre de forma a permitir que
os projetos possam facilmente ligar seus repositórios e páginas à
ferramenta.

\subsection{Histórias}
\label{subsec:stories}

Com relação ao sistema de planejamento, XP sugere que os requisitos
deveriam ser coletados em cartões de histórias. O objetivo disto é
reduzir a quantidade de esforço necessário para descobrir qual é o
próximo passo a ser tomado e tornar fácil modificar essas prioridades
ao longo do tempo. Projetos de software livre normalmente guardam seus
requisitos em sistemas de rastreamento de erros. Quando se identifica
a falta de uma funcionalidade, cadastra-se um erro que deveria ser
corrigido e as discussões e sugestões de mudanças são enviadas para
aquele ``erro''. O problema com essa abordagem é que mudar a
prioridade desses ``erros'' e organizar um planejamento consome muito
tempo e se basea em fatos que podem mudar com o tempo (tal como ``essa
versão deveria resolver erros com prioridade acima de 8''). Também é
muito difícil obter uma visão geral dos requisitos.

Descobrir as principais prioridades para a equipe rapidamente e ser
capaz de mudar essas prioridades de acordo com o \emph{feedback} é uma
das chaves para desenvolver software funcional. Para poder atingir
esse objetivo, uma ferramenta deveria ser desenvolvida para permitir
que erros sejam vistos como objetos móveis num quadro de planejamento
de versão. Para permitir que a comunidade envolvida possa colaborar
com seu conhecimento, a ferramenta deveria apresentar a prioridade do
erro assim como seu conteúdo de uma forma similar ao dos artigos da
Wikipedia \cite{Surowiecki2004,Tapscott2006,Benkler2006}.

\subsection{Retrospectiva}
\label{subsec:retrospect}

Essa prática sugere que a equipe deveria se juntar num ambiente físico
periodicamente para discutir o andamento do projeto. Existem dois
problemas nessa prática em equipes de software livre. O primeiro é de
que todos os membros da equipe devem estar presentes ao mesmo tempo no
mesmo lugar. O segundo é fazer com que a equipe interaja de forma
coletiva para apontar os problemas e as soluções que surgiram durante
o período avaliado. A forma mais comum para ajudar os participantes a
realizar esse trabalho é apresentar uma linha temporal e pedir para
que eles façam anotações sobre os eventos que ocorreram nesse
período. Isso os ajuda a relembrar os acontecimentos e entender porque
as coisas aconteceram da forma que aconteceram.

Quando a equipe está co-locada, basta juntar a equipe numa sala de
reunião com uma linha do tempo grande na parede e distribuir papéis
coloridos que eles possam colar na linha. A sugestão para equipes de
software livre é desenvolver uma ferramenta baseada na Internet para
permitir que essas anotações sejam feitas numa linha do tempo virtual
associada aos código fonte. Desta forma, mensagens de integração de
código poderiam conter a anotação que seriam automaticamente exibidos
na linha do tempo. Além disso, a equipe poderia anotar a linha do
tempo de forma assíncrona para permitir comentários posteriores. O
líder da equipe poderia ocasionalmente gerar um relatório para todos
os membros da equipe além de exibir a linha do tempo no ambiente
informativo.

\subsection{Papo em Pé}
\label{subsec:stand-up}

Papos em pé, originalmente sugeridos em Scrum, pede que toda a equipe
se junte e cada membro explique rapidamente o que eles têm feito e
pretendem fazer a seguir. Essa prática compartilha dos mesmos
problemas da retrospectiva. Ela envolve reunir a equipe ao mesmo
tempo. Muitos projetos de software livre usam canais de IRC
(\emph{Internet Relay Chat}) para resolverem parcialmente esse
problema e para centralizar as discussões durante o
desenvolvimento. Apesar disso não garantir que todos sabem o que cada
um está fazendo, ajuda a sincronizar o trabalho.

Para garantir que os membros obtenham a informação necessária, a
sugestão é de que a comunicação que acontece nesses canais IRC seja
salva e exibida aos usuários que se acabam de se conectar. Também
deveria ser possível permitir que os usuários deixem anotações a
partir desse canal para o sistema de rastreamento de erros assim como
mensagens para outros contribuidores. No canal IRC, esse tipo de
solução normalmente é implementada por um robô que deveria estar
ligado à incubadora do projeto que contém as ferramentas previamente
sugeridas.

Tendo delineado as práticas em que há uma possibilidade de
aproveitamento em cada uma das respectivas comunidades, pode-se traçar
o plano de trabalho proposto. O próximo capítulo (Capítulo
\ref{cap:perspectivas}) apresenta o trabalho que se pretende elaborar
assim como as metas para sua realização.
