%% ------------------------------------------------------------------------- %%
\chapter{Definições}
\label{cap:definicoes}

Para poder falar sobre software livre e métodos ágeis, é necessário,
primeiro, definir o que deve ser entendido por estas palavras. Métodos
ágeis são definidos na Seção \ref{sec:agile-def} enquanto a definição
de software livre, mais controversa, é dada na Seção \ref{sec:os-def}.

\section{Definição de Métodos Ágeis}
\label{sec:agile-def}

Este trabalho considerará que qualquer método de engenharia de
software que siga os princípios do manifesto ágil
\cite{AgileManifesto} é um método ágil. O interesse principal do
trabalho está ligado aos métodos mais conhecidos, como Programação
Extrema (XP) \cite{XP02}, Scrum \cite{Schwaber2004} e a família
Crystal \cite{Cockburn2002}. Mas também serão mencionadas algumas
ideias relacionadas à ``filosofia'' \emph{Lean} \cite{Ohno1998} e sua
aplicação ao desenvolvimento de software \cite{Poppendieck2005}.

\section{Definição de Software Livre}
\label{sec:os-def}

Os termos ``Software de Código Aberto'' e ``Software Livre'' serão
considerados os mesmos neste trabalho apesar de terem diferenças
importantes em seus contextos específicos \cite[Ch. 1, Free Versus
Open source]{Fogel2005}. Projetos serão considerados de código aberto
(ou livres) se seu código fonte estiver disponível e puder ser
modificado por qualquer pessoa com o conhecimento técnico necessário
sem consentimento prévio do autor original e sem encargos. Note que
esta definição está mais próxima da ideia de software livre do que da
de código aberto.

Outra restrição será de que projetos iniciados e controlados por uma
empresa não serão considerados livres. Isto porque projetos
controlados por empresas onde seja disponibilizado o código fonte e/ou
sejam aceitas colaborações externas podem ser desenvolvidos com
qualquer método de engenharia de software. Basta que a empresa obrigue
seus funcionários a seguir determinado método. Alguns incluem práticas
que atraem contribuições externas outros distribuem apenas o trabalho
escolhido aos membros da equipe. De qualquer forma, a empresa controla
sua própria equipe e mantém o controle sobre o desenvolvimento
independentemente de colaborações.

No entanto, projetos livres baseados em comunidades de empresas podem
ser caracterizados como projetos de software livre se não existir um
contrato que force cada empresa a dedicar uma determinada quantidade
de horas de trabalho para o projeto. Caem neste caso o Eclipse com a
\emph{Eclipse Foundation} que, apesar de ter sido iniciado e pela IBM,
agrega diversas empresas parceiras e o Java$^{TM}$ com o \emph{Java Community
  Process} que permite que a comunidade tome decisões sobre o
desenvolvimento da linguagem apesar da Sun ser proprietária da
marca. Esses contextos se assemelham ao de um desenvolvedor ou uma
equipe central trabalhando em conjunto com indivíduos ou equipes de
forma voluntária e, por isso, podem ser considerados software livre
conforme o contexto deste texto.

Considerando esta definição, é importante caracterizar as pessoas
envolvidas em tais projetos. Em 2002, o \emph{FLOSS Project}
\cite{FlossProject} publicou um relatório sobre uma pesquisa realizada
com contribuidores de projetos de software livre. Os dados coletados
\cite{FlossStats} mostram que 78.77\% dos contribuidores têm emprego
(pergunta 42) e que apenas 50.82\% da comunidade de software livre são
programadores enquanto 24.76\% não ganham a maioria de suas rendas com
desenvolvimento de software (pergunta 10). Além desses resultados, a
pesquisa apresenta o fato de 78.78\% dos colaboradores considerarem
suas tarefas em projetos livres mais prazerosas (pergunta 22.2) do que
suas atividades regulares. 42.30\% também consideram seus projetos de
software livre mais organizados que seus projetos profissionais
(pergunta 22.4). Como conclusão, poderia-se afirmar que contribuidores
de software livre tendem a considerar suas atividades nos projetos são
mais prazerosas e eficientes. Esses sentimentos sobre essas atividades
podem estar ligados à liberdade no sistema de desenvolvimento, no qual
não há nenhum processo pesado.

Por pesado, entende-se um processo no qual é muito importante
documentar rigorosamente as decisões tomadas e a maneira na qual
atingiu-se essa decisão. Tipicamente, estes processos contam com uma
importante fase de planejamento de forma a garantir que os documentos
que explicam a tomada de decisão sejam úteis e apresentem análises das
várias possibilidades. A palavra pesada é usada como oposição aos
processos ditos leves nos quais espera-se que o esforço de
documentação seja investido no código. % TODO é?

Outra pesquisa \cite{Reis2003} apontou que 74\% dos projetos de
software livre tem equipes com até 5 pessoas e que 62\% dos
contribuidores trabalham entre si pela Internet e nunca se conheceram
fisicamente. Portanto, é crítico para esses projetos que o processo de
desenvolvimento esteja adequado a essas características e não se torne
um fardo para o trabalho voluntário.

Tendo deixado claro o significado de Métodos Ágeis e Software Livre
neste texto, o capítulo seguinte (Capítulo \ref{cap:foss})
aborda as semelhanças e diferenças entre o desenvolvimento de Software
Livre e os conceitos de Métodos Ágeis.
