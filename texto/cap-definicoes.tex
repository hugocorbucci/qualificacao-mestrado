%% ------------------------------------------------------------------------- %%
\chapter{Defini��es}
\label{cap:definicoes}

In order to start talking about OS and agile methods, it is necessary
to first define what is understood by such words. Agile methods are
defined in Section \ref{sec:agile-def} while the OS definition,
more controversial, is given in Section \ref{sec:os-def}.

\section{Agile methods definition}
\label{sec:agile-def}

This work will consider that any software engineering method that
follows the principles of the agile manifesto \cite{AgileManifesto} is
an agile method. Focus will be given on the most known methods, such
as eXtreme Programming (XP) \cite{XP02}, Scrum \cite{Schwaber2004} and
the Crystal family \cite{Cockburn2002}. Closely related ideas will
also be mentioned from the wider Lean philosophy \cite{Ohno1998} and
its application to software development \cite{Poppendieck2005}.

\section{Open source definition}
\label{sec:os-def}

The terms ``Open source software'' and ``Free software'' will be
considered the same in this work although they have important
differences in their specific contexts \cite[Ch. 1, Free Versus Open
source]{Fogel2005}. Projects will be considered to be open source (or
free) if their source code is available and modifiable by anyone with
the required technical knowledge, without prior consent from the
original author and without any charge. Note that this definition is
closer from the free software idea than the open source one.

Another restriction will be that projects started and controlled by a
company do not fit in this definition of OS. That is because projects
controlled by companies, whether they have a public source code and
accept external collaboration or not, can be run with any software
engineering method since the company can enforce it to its
employees. Some methods will work better to attract external
contributions but the company still controls its own team and can
maintain the software without external collaboration.

Considering this definition, it is important to characterize the
people involved in such projects. In 2002, the FLOSS Project
\cite{FlossProject} published a report about a survey they conducted
regarding FOSS contributors. Their collected data \cite{FlossStats}
shows that 78.77\% of the contributors are employed or self-employed
(question 42) and that only 50.82\% of the OS community are software
developers while 24.76\% do not earn their main income with software
development (question 10).  In addition to those results, the survey
presents the fact that 78.78\% of the collaborators consider their OS
tasks more joyful (question 22.2) than their regular activities and
42.3\% also consider them better organized (question 22.4). As a
conclusion, we could say that OS contributors perceive their
activities both pleasurable and effective. Having those feelings about
the activities might be linked to the freedom in the development
system, where there is no heavy process attached.

Another survey \cite{Reis2003} points out that 74\% of open source
projects have teams with up to 5 people and 62\% of the contributors
work with each other over the Internet and never met physically.  It
is therefore critical for those projects to have an adequate software
process that fits those characteristics and is not a burden on the
volunteer work.