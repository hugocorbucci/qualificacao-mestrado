%% ------------------------------------------------------------------------- %%
\chapter{Desenvolvimento de Software Livre é Ágil?}
\label{cap:foss}

Comunidades de software livre poderiam praticamente ser consideradas
ágeis e, de fato, elas foram consideradas como tal por Martin Fowler
na sua primeira versão de \emph{``The New Methodology''}
\cite{Fowler00orig}. Os métodos descritos por Eric Raymond em
\emph{``The Cathedral and the Bazaar''} \cite{Raymond1999} pecam por
não serem definidos mais precisamente. Apesar disso, diversas idéias
poderiam ser relacionadas ao manifesto ágil \cite{AgileManifesto}. As
próximas quatro seções apresentam a relação entre as atitudes
encontradas na maioria das comunidades de software livre e cada um dos
quatro princípios enunciados pelo manifesto. A quinta seção irá
recapitular os pontos nos quais os projetos de software livre poderiam
tornar-se mais ágeis.

\section{Indivíduos e interações são mais importantes que processos e
  ferramentas}
\label{sec:first-princ}

Várias pesquisas relacionadas a desenvolvimento de software livre
apresentam uma quantidade razoável de ferramentas usadas por
desenvolvedores para manter a comunicação entre os membros da
equipe. Reis \cite{Reis2003} mostra que 65\% dos projetos analisados
usam programas de controle de versão, a página na Internet do projeto
e listas de correio eletrônico como as principais ferramentas de
comunicação entre os usuários do programa e a equipe de
desenvolvimento. \textbf{Os processos e ferramentas} são, no entanto,
apenas um meio de atingir um objetivo: garantir um ambiente estável e
acolhedor para a criação do programa de forma colaborativa.

Apesar dos negócios baseados em software livre estarem crescendo, a
essência da comunidade ao redor do programa é de manter
\textbf{indivíduos que interajam} de forma a produzir o que lhes
interessa. As ferramentas apenas permitem isso. Nessas comunidades,
interações são normalmente relacionadas a colaboração para o código
fonte e a elaboração de documentação, independente do modelo de
negócios. Essas atividades são responsáveis por dirigir o processo e
modificar as ferramentas para que elas cumpram melhor as necessidades
da comunidade.

%FIXME Continuar a traduzir

\section{Working software over comprehensive documentation}
\label{sec:second-princ}

According to Reis \cite{Reis2003}, 55\% of the OS projects update and
revise their documentation frequently and 30\% maintain documents that
explain how parts of it work or how is it organized. Those results
show that user documentation is considered important but is not the
final goal of the projects. On the other hand, requirements are
described as bugs and stored in a bug tracking system since they
demand software modifications.

More recently, Oram \cite{Oram2007} presented the results of a survey
conducted by O'Reilly showing that free documentation is increasingly
being written by volunteers. It means that \textbf{comprehensive
  documentation} grows with the community around the \textbf{working
  software}, as users encounter problems to complete a specific
action. According to Oram's work, the most important reasons for
contributors to write documentation is for their personal growth or to
improve the community. This motivation explains why OS documentation
usually comprehend the most common problems and explain how to use the
most frequently used features but are faulty to provide details about
less popular features.

\section{Customer collaboration over contract negotiation}
\label{sec:third-princ}

\textbf{Contract negotiation} is still only a problem to very few open
source projects since a huge number of them do not involve
contracts. On the other hand, those involving contracts are usually
based on a service concept in which the customer hires a programmer or
company to develop a certain feature for a small amount of
time. Although this business model does not ensure that the customer
will collaborate, it may shorten the time between conversations,
therefore improving feedback and reducing the strength of rigid
contracts.

The key point here is that collaboration is the basis of OS
projects. The customer is involved as much as he desires to
be. \textbf{Customers can collaborate} but they are not especially
encouraged or forced to do so. This might be related to the small
amount of experience this communities has with customer
relationships. However, several successful projects rely on fast
answers to features demanded by users. In this case user (or customer)
collaboration allied with responsiveness are specially powerful.

\section{Responding to change over following a plan}
\label{sec:fourth-princ}

OS projects tend to have a plan of milestones or releases. Several
projects only count with short term plans. When long term plans exist,
they are not the main guidelines followed by the developer team but
only goals sought without any pressure to be
met. % TODO: Achar referencia

Being too demanding about \textbf{following a plan} can drag a whole
project down in the OS world. The main reason is the highly
competitive environment of this universe where only the best projects
survive. The \textbf{ability of each project to adapt and respond to
  changes} is crucial to determine those who survive. No marketing
campaign or business deal can save a project from abandonment if it
cannot compete with a newcomer that adapts more quickly to user needs.

\section{What is missing on open source?}
\label{sec:os-summary}

Although several points of the agile manifesto are followed within OS
communities, there is no such thing as an OS method. Raymond's
description \cite{Raymond1999} is a great example of how the process
can work but it does not discriminate guidelines and practices to be
followed. If a careful description of an open source process was
written, it should merge the ideas presented by Raymond with a process
definition. This process would then follow the same selection rules as
the projects. If appropriate, its adoption would then spread around
the community improving and correcting it over time and creating the
missing tools.

Communities created around FOSS projects involve users, developers,
and sometimes even clients working together to craft the best software
possible. The absence of such community around a program usually
denounces a recent project or one that is dying. Those signs mean that
the development team must be very attentive to its software community
which shows the health of the project. Nowadays, concerns related to
this aspect of FOSS development are not specifically considered by the
most known agile methods.

