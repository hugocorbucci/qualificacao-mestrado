%% ------------------------------------------------------------------------- %%
\chapter{Is Open source Agile?}
\label{cap:foss}

OS communities could almost be considered agile and they indeed were
by Martin Fowler in his first version of ``The New Methodology''
\cite{Fowler00orig}. The methods that Eric Raymond describes in ``The
Cathedral and the Bazaar'' \cite{Raymond1999} lack a more precise
definition but several ideas could be related to the agile manifesto
\cite{AgileManifesto}. The next four subsections will discuss the
relation of OS and each of the four principles of the manifesto and
the fifth one will summarize points where OS could improve towards
agility.

\section{Individuals and interactions over processes and tools}
\label{sec:first-princ}

Several researches regarding OS software development present a
reasonable amount of tools used by the developers to maintain
communication between their members. Reis \cite{Reis2003} shows that
65\% of the studied projects use version control software, the project
website and mailing lists as the most used tools to communicate with
the users and in the team. \textbf{The processes and tools} are,
however, just a mean to achieve a goal: ensuring a stable and
welcoming environment to create software collaboratively.

Although OS businesses are growing stronger, the very essence of the
community around the software is to have \textbf{individuals that
  interact} in order to produce what interests them. The tools only
permit that. In those communities, interaction is usually related to
source code collaboration and documentation elaboration regardless of
the business model. Those activities are responsible for driving the
whole process and modifying the tools to better fit their needs.

\section{Working software over comprehensive documentation}
\label{sec:second-princ}

According to Reis \cite{Reis2003}, 55\% of the OS projects update and
revise their documentation frequently and 30\% maintain documents that
explain how parts of it work or how is it organized. Those results
show that user documentation is considered important but is not the
final goal of the projects. On the other hand, requirements are
described as bugs and stored in a bug tracking system since they
demand software modifications.

More recently, Oram \cite{Oram2007} presented the results of a survey
conducted by O'Reilly showing that free documentation is increasingly
being written by volunteers. It means that \textbf{comprehensive
  documentation} grows with the community around the \textbf{working
  software}, as users encounter problems to complete a specific
action. According to Oram's work, the most important reasons for
contributors to write documentation is for their personal growth or to
improve the community. This motivation explains why OS documentation
usually comprehend the most common problems and explain how to use the
most frequently used features but are faulty to provide details about
less popular features.

\section{Customer collaboration over contract negotiation}
\label{sec:third-princ}

\textbf{Contract negotiation} is still only a problem to very few open
source projects since a huge number of them do not involve
contracts. On the other hand, those involving contracts are usually
based on a service concept in which the customer hires a programmer or
company to develop a certain feature for a small amount of
time. Although this business model does not ensure that the customer
will collaborate, it may shorten the time between conversations,
therefore improving feedback and reducing the strength of rigid
contracts.

The key point here is that collaboration is the basis of OS
projects. The customer is involved as much as he desires to
be. \textbf{Customers can collaborate} but they are not especially
encouraged or forced to do so. This might be related to the small
amount of experience this communities has with customer
relationships. However, several successful projects rely on fast
answers to features demanded by users. In this case user (or customer)
collaboration allied with responsiveness are specially powerful.

\section{Responding to change over following a plan}
\label{sec:fourth-princ}

OS projects tend to have a plan of milestones or releases. Several
projects only count with short term plans. When long term plans exist,
they are not the main guidelines followed by the developer team but
only goals sought without any pressure to be
met. % TODO: Achar referencia

Being too demanding about \textbf{following a plan} can drag a whole
project down in the OS world. The main reason is the highly
competitive environment of this universe where only the best projects
survive. The \textbf{ability of each project to adapt and respond to
  changes} is crucial to determine those who survive. No marketing
campaign or business deal can save a project from abandonment if it
cannot compete with a newcomer that adapts more quickly to user needs.

\section{What is missing on open source?}
\label{sec:os-summary}

Although several points of the agile manifesto are followed within OS
communities, there is no such thing as an OS method. Raymond's
description \cite{Raymond1999} is a great example of how the process
can work but it does not discriminate guidelines and practices to be
followed. If a careful description of an open source process was
written, it should merge the ideas presented by Raymond with a process
definition. This process would then follow the same selection rules as
the projects. If appropriate, its adoption would then spread around
the community improving and correcting it over time and creating the
missing tools.

Communities created around FOSS projects involve users, developers,
and sometimes even clients working together to craft the best software
possible. The absence of such community around a program usually
denounces a recent project or one that is dying. Those signs mean that
the development team must be very attentive to its software community
which shows the health of the project. Nowadays, concerns related to
this aspect of FOSS development are not specifically considered by the
most known agile methods.

