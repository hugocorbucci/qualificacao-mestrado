%% ------------------------------------------------------------------------- %%
\chapter{Conclus�es}
\label{cap:conclusoes}

%------------------------------------------------------
\section{Considera��es Finais} 


In this preliminary work we have shown several evidences that a
synergy between agile methods and OS can improve software development
on FOSS projects. Several projects already adopt some agile techniques
to be more responsive to users but a complete description of a method
that considers all FOSS factors would surely increase adoption in
those communities. On the other hand, solving the problem is a
challenge that would consolidate agile methods to a distributed
environment relying on a large user community.

As part of this work, two surveys are planned. One to be conducted at
FISL (International Free Software Forum) 2009 to understand how much
OS developers and enthusiasts know about agile methods and what keeps
them from using them. The other one to be conducted at Agile 2009 will
try to discover how involved is the agile community with OS
development. Both surveys will be used to provide a deeper
understanding of the interaction between both communities and how to
improve it.


%------------------------------------------------------
\section{Sugest�es para Pesquisas Futuras} 

\begin{itemize}
	\item Elaborar uma metodologia direcionada para o
          desenvolvimento de software livre com equipes distribu�das
          em um ambiente �gil.
\end{itemize}
