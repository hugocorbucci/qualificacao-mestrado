%% ------------------------------------------------------------------------- %%
\chapter{Próximos Passos}
\label{cap:perspectivas}

O trabalho apresentado dá indícios de que o estado atual das
comunidades de software livre e de métodos ágeis pode ser melhorado
pela união dos conhecimentos e soluções encontradas em cada um destes
ambientes. Este projeto pretende dar alguns dos passos necessários
para essa união que estão listados na Seção \ref{sec:atividades} com
as metas apresentadas na Seção \ref{sec:cronograma}. Por fim, a Seção
\ref{sec:sugestoes} indica algumas das atividades que podem trazer
benefícios a essa união mas que não serão cobertas por este trabalho.

\section{Atividades a Serem Realizadas}
\label{sec:atividades}

Após revisão das pesquisas disponíveis nos apêndices \ref{ape:OS} e \ref{ape:MA}, ambas pesquisas 
serão disponibilizadas na Internet e
divulgadas em grandes eventos de cada uma das comunidades. Os
principais objetivos seriam o décimo Fórum Internacional de Software
Livre (FISL 10.0) do lado da comunidade de software livre e a Agile
2009 do lado da comunidade de métodos ágeis. Se possível, outros
vetores de divulgação também serão usados para tentar obter o maior
número de respostas possíveis.

As pesquisas usarão um sistema a ser determinado que permita a
identificação dos usuários para evitar o preenchimento da pesquisa por
robôs assim como duplicatas. Esse sistema é uma das atividades que
deverão ser desenvolvidas como parte desse trabalho. O sistema deverá
contar com as funcionalidades necessárias para a elaboração das
pesquisas assim como o preenchimento das respostas de acordo com a
descrição anterior. Além disso, ele deverá fornecer um módulo de
relatório que permita visualizar os dados gerais facilmente.

Os resultados das pesquisas serão usados na identificação dos
principais problemas de comunicação encontrados por cada uma das
comunidades na adoção das técnicas da outra. Quando esses problemas
forem identificados, espera-se criar um conjunto de pequenos
aplicativos que ajudem a resolver os problemas encontrados. A ideia é
de que esses aplicativos sejam pequenos programas em Javascript que
permitam que o usuário mantenha dados atualizados sobre seu projeto
com consultas ao seu repositório de código.

Tendo em mãos esses pequenos aplicativos e os problemas de comunicação
que eles pretendem resolver, espera-se fazer uma análise deles com
relação à sua adequação com o Modelo de Maturidade para Software Livre
(OMM - \emph{Open Source Maturity Model}\footnote{Primeira versão do
  modelo disponível em:
  http://www.qualipso.org/sites/default/files/A6.D1.6.3CMM-LIKEMODELFOROSS.pdf
  - Último acesso 29/05/2009}) que está sendo desenvolvido pelo
projeto QualiPSo. A análise será baseada na taxonomia proposta nas
recomendações do projeto Qualipso \cite{Malheiros2009} para aumentar a
qualidade de projetos livres. A ideia é caracterizar as propostas de
soluções oferecidas pelas ferramentas, de acordo com a taxonomia
descrita, de forma a evidenciar os benefícios que cada ferramenta pode
trazer.

Com essa caracterização será mais fácil verificar quais pontos
descritos pelo OMM são abordados pelas ferramentas e quais ainda
precisam ser tratados. Caso o trabalho aborde uma quantidade razoável
de pontos, ele poderá ser estendido para propor uma adaptação de algum
processo ágil com o uso das ferramentas desenvolvidas que se adeqúe ao
primeiro nível do modelo de maturidade citado.

Por fim, os resultados das pesquisas serão condensados em um relatório
que será anexado às ferramentas descritas anteriormente e será
elaborada uma dissertação com a análise desses resultados e com as
conclusões obtidas do trabalho de pesquisa e de desenvolvimento e a
análise do modelo de maturidade do projeto QualiPSo.

\section{Metas para as Atividades}
\label{sec:cronograma}

Com relação às pesquisas, a meta é de que a pesquisa direcionada à
comunidade de software livre esteja disponíveis \emph{online} até o
dia 15 de Junho de 2009. Desta forma, sua divulgação poderia ser
realizada durante a semana que precede o FISL 10.0. Espera-se manter a
pesquisa disponível por, pelo menos, 2 meses. Desta forma, espera-se
colher os dados relativos a essa pesquisa por volta do dia 15 de
Agosto de 2009.

Já a pesquisa direcionada à comunidade de métodos ágeis deveria estar
disponível a partir de dia 1 de Agosto de 2009 para ser divulgada
durante as duas semanas que antecedem a Agile 2009. Mantendo o mesmo
intervalo de tempo para coletar as respostas, os resultados seriam
coletados por volta do dia 1 do Outubro 2009.

A avaliação das pesquisas será realizada de forma independente num
primeiro momento durante as duas semanas que seguem a coleta dos
resultados. Em seguida, os dados das pesquisas serão cruzados para
tentar identificar as respostas provindas dos mesmos indivíduos nas
duas comunidades. Desta forma, espera-se concluir a avaliação dos
resultados das pesquisas por volta do dia 1 de Novembro de 2009.

O desenvolvimento das ferramentas depende apenas parcialmente dos
resultados obtidos na análise conjunta das pesquisas e por isso pode
ter início antes dos resultados completos serem coletados. Desta
forma, espera-se montar a infra-estrutura para desenvolvimento e
elaboração das ferramentas enquanto a pesquisa ocorre. Espera-se
dedicar apenas 3 meses de desenvolvimento após o término de ambas
pesquisas e da avaliação dos resultados. Com isso, a meta é finalizar
o desenvolvimento das ferramentas até o dia 1 de fevereiro 2010.

A dissertação decorrente do trabalho deverá evoluir conforme o
trabalho avança e, por isso, deveria acompanhar o desenvolvimento da
ferramenta com um período de um mês para revisão e finalização do
texto. Com isso, a dissertação deveria estar pronta em março de 2010
como mostra a tabela \ref{tab:crono}.

\begin{table}
  \centering
  \begin{tabular}{|l|c|c|c|c|c|c|c|c|c|c|c|}
    \hline
    & 5/9 & 6/9 & 7/9 & 8/9 & 9/9 & 10/9 & 11/9 & 12/9 &
    1/10 & 2/10 & 3/10\\
    \hline
    Pesquisa Software Livre & X & X & X & & & & & & & &\\
    \hline
    Pesquisa Métodos Ágeis & & & & X & X & X & & & & &\\
    \hline
    Análise  & & & & X & X & X & X & & & &\\
    \hline
    Desenvolvimento & & & X & X & X & X & X & X & X & X & \\
    \hline
    Dissertação & & & & X & X & X & X & X & X & X & X \\
    \hline
  \end{tabular}
  \caption{Prazos esperados para as atividades}
  \label{tab:crono}
\end{table}

% ------------------------------------------------------
\section{Sugestões para Pesquisas Futuras}
\label{sec:sugestoes}

Este trabalho não irá se aprofundar na elaboração de um método de
desenvolvimento que se adere aos princípios ágeis com foco na produção
de software livre mas este, com certeza, é um tópico interessante para
futuras pesquisas. Imaginar uma forma de balancear o desenvolvimento e
a resposta às mudanças descobertas com a manutenção da comunidade e
atração de colaboradores pode trazer grandes avanços à maneira com que
os métodos ágeis lidam com seus clientes.

Outro trabalho interessante seria o de tentar analisar os impactos que
a cultura associada a uma linguagem de programação tem nos projetos
elaborados nessa linguagem. A cultura de testes automatizados tem
crescido muito nos últimos tempos mas algumas comunidades como a de
Ruby e Python têm tido um destaque nesse quesito. Isso vem da forma
como a linguagem é apresentada ou da comunidade que a cerca. Com isso,
como uma determinada comunidade (digamos a de métodos ágeis) pode
influenciar os projetos criados simplesmente criando boas ferramentas
para suas práticas nas linguagens usadas.
