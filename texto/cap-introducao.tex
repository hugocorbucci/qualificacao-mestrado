%% ------------------------------------------------------------------------- %%
\chapter{Introdu��o}
\label{cap:introducao}

%% ------------------------------------------------------------------------- %%
\section{Considera��es Preliminares}
\label{sec:consideracoes_preliminares}

Typical Open Source (OS) projects (to be defined on Section
\ref{sec:os-def}) usually receive the collaboration of many
geographically distant people \cite{Dempsey1999} and are organized
around a leader which is the top of the hierarchical structure in the
group. At first glance, this argument could indicate that such
projects are not candidates for the use of agile methods since some
basic values seem to be missing. For example, the distance and
diversity separating developers deteriorates communication, a very
important value within agile methods. However, most open source
projects share some principles with the agile manifesto
\cite{AgileManifesto}. Being ready for changes, working with
continuous feedback, delivering real features, respecting
collaborators and users and facing challenges are expected attitudes
from agile developers naturally found in the Free and Open Source
Software (FOSS) communities.

%\index{genoma!projetos}
% index permite acrescentar um item no indice remissivo

%% ------------------------------------------------------------------------- %%
\section{Objetivos}
\label{sec:objetivo}

During a workshop \cite{OOPSLA07} about ``No Silver Bullets''
\cite{Brooks1987} held at OOPSLA 2007, agile methods and OS software
development were mentioned as two failed silver bullets having both
brought great benefit to the software community. During the same
workshop the question was raised whether the use of several failed
silver bullets simultaneously could not raise production levels by an
order of magnitude. This is an attempt to suggest one of those merges
to partially tackle software development problems.

%% ------------------------------------------------------------------------- %%
\section{Contribui��es}
\label{sec:contribucoes}

As principais contribui��es deste trabalho est�o discriminadas abaixo:

\begin{itemize}
\item Pesquisas com a comunidade de software livre
\item Pesquisas com a comunidade de m�todos �geis
\item Elabora��o do conjunto de ferramentas ``Agile Portlets''
  dispon�veis em \url{http://launchpad.net/project/agile-portlets/}.
\end{itemize}

%% ------------------------------------------------------------------------- %%
\section{Organiza��o do Trabalho}
\label{sec:organizacao_trabalho}

The topics discussed in this work consider only a subset of projects
that are said to be agile or OS. Chapter \ref{cap:definicoes} presents
the definitions used in our work.  Chapter \ref{cap:foss} will present
some aspects of major OS communities that could be improved with agile
practices and principles. The next chapter (Chapter \ref{cap:agile})
will focus on problems that arise when using agile methods with
distributed and large teams which have somehow already been addressed
in OS development. Finally, Chapter \ref{cap:conclusion} will
summarize our current work and present our future tasks.
