%% ------------------------------------------------------------------------- %%
\chapter{Introdução}
\label{cap:introducao}

%% ------------------------------------------------------------------------- %%
\section{Considerações Preliminares}
\label{sec:consideracoes_preliminares}

Projetos de Software Livre (SL) típicos (que serão definidos no
Capítulo \ref{cap:definicoes}) normalmente recebem a colaboração de
muitas pessoas geograficamente distantes \cite{Dempsey1999} e se
organizam ao redor de um líder que está no topo da estrutura do grupo.

Num primeiro momento, este argumento poderia indicar que esse tipo de
projeto não é candidato para o uso de métodos ágeis de desenvolvimento
de software já que alguns valores essenciais parecem ausentes. Por
exemplo, a distância entre os desenvolvedores e a diversidade entre
suas culturas dificulta enormemente a comunicação que é um dos
principais valores de métodos ágeis. No entanto, a maioria dos projetos
de software livre compartilham alguns princípios enunciados no
manifesto ágil \cite{AgileManifesto}. Adaptação a mudanças, trabalhar
com \emph{feedback} contínuo, %TODO Traduzir
entregar funcionalidades reais, respeitar colaboradores e usuários e
enfrentando desafios são atitudes esperadas de desenvolvedores de
métodos ágeis que são naturalmente encontradas em comunidades de
Software Gratuito, Livre e Aberto (FLOSS - \emph{Free, Libre and Open
  Source Software}). %TODO Faz assim mesmo?

% \index{genoma!projetos} index permite acrescentar um item no indice
% remissivo

%% ------------------------------------------------------------------------- %%
\section{Objetivos}
\label{sec:objetivo}

Durante um \emph{workshop} \cite{OOPSLA07} sobre \emph{``No Silver
  Bullets''} \cite{Brooks1987} na conferência OOPSLA 2007, métodos
ágeis e software livre foram mencionados como duas balas de prata
fracassadas que trouxeram grandes benefícios à comunidade de software
apesar de não terem resolvido o problema. Durante o mesmo
\emph{workshop}, perguntou-se se o uso de várias balas de prata
fracassadas não poderia fazer o papel de uma bala de prata real, isto
é, aumentar de uma ordem de magnitude os níveis de produção. Este
trabalho é uma tentativa de sugerir uma dessas uniões para diminuir os
problemas de desenvolvimento de software.

%% ------------------------------------------------------------------------- %%
\section{Contribuições}
\label{sec:contribucoes}

As principais contribuições deste trabalho estão discriminadas abaixo:

\begin{itemize}
\item Pesquisas com a comunidade de software livre
\item Pesquisas com a comunidade de métodos ágeis
\item Elaboração do conjunto de ferramentas ``Agile Portlets''
  disponíveis em \url{http://launchpad.net/project/agile-portlets/}.
\end{itemize}

%% ------------------------------------------------------------------------- %%
\section{Organização do Trabalho}
\label{sec:organizacao_trabalho}

Os tópicos apresentados nesse trabalho consideram apenas um
subconjunto de projetos que são ditos ágeis ou software livre. O
Capítulo \ref{cap:definicoes} apresenta as definições usadas nesse
trabalho. O Capítulo \ref{cap:foss} apresenta alguns aspectos da
maioria das comunidades de software livre que poderiam ser melhorados
com práticas ou princípios ágeis. O capítulo seguinte (Capítulo
\ref{cap:agile}) aborda alguns problemas que surgem ao usar métodos
ágeis com equipes grandes e distribuídas que já foram endereçados nas
comunidades de software livre. Por fim, o Capítulo
\ref{cap:conclusoes} resume o trabalho realizado e apresenta os planos
para o futuro.
