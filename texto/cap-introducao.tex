%% ------------------------------------------------------------------------- %%
\chapter{Introdução}
\label{cap:introducao}

%% ------------------------------------------------------------------------- %%
\section{Considerações Preliminares}
\label{sec:consideracoes_preliminares}

Projetos de Software Livre (SL) típicos (que serão definidos no
Capítulo \ref{cap:definicoes}) normalmente recebem a colaboração de
muitas pessoas geograficamente distantes \cite{Dempsey1999} e se
organizam ao redor de um ou mais líderes que estão no topo da
estrutura do grupo.

Num primeiro momento, este argumento poderia indicar que esse tipo de
projeto não é candidato para o uso de métodos ágeis de desenvolvimento
de software já que alguns valores essenciais parecem ausentes. Por
exemplo, a distância entre os desenvolvedores e a diversidade entre
suas culturas dificulta enormemente a comunicação que é um dos
principais valores de métodos ágeis. No entanto, a maioria dos
projetos de software livre compartilham alguns princípios enunciados
no manifesto ágil \cite{AgileManifesto}. Adaptação a mudanças,
trabalhar com \emph{feedback} contínuo, entregar funcionalidades
reais, respeitar colaboradores e usuários e enfrentando desafios são
atitudes esperadas de desenvolvedores de métodos ágeis que são
naturalmente encontradas em comunidades de Software Gratuito, Livre e
Aberto (FLOSS - \emph{Free, Libre and Open Source Software}).

% \index{genoma!projetos} index permite acrescentar um item no indice
% remissivo

%% ------------------------------------------------------------------------- %%
\section{Objetivos}
\label{sec:objetivo}

Durante um \emph{workshop} \cite{OOPSLA07} sobre \emph{``No Silver
  Bullets''} \cite{Brooks1987} na conferência OOPSLA 2007, métodos
ágeis e software livre foram mencionados como duas balas de prata
fracassadas que trouxeram grandes benefícios à comunidade de software
apesar de não terem resolvido de forma completa os problemas ligados
ao desenvolvimento de software. Durante o mesmo \emph{workshop},
perguntou-se se o uso de várias balas de prata fracassadas não poderia
fazer o papel de uma bala de prata real, isto é, aumentar em uma ordem
de magnitude os níveis de produção.

Uma pesquisa realizada num encontro que reuniu aproximadamente 200
pessoas interessadas em métodos ágeis, foi realizada uma pesquisa para
descobrir se a associação entre métodos ágeis e software livre é
comum. A pesquisa (disponível no Apêndice \ref{ape:EA}) foi realizada
em papel e entregue a todos os participantes do encontro no início do
evento e recolhida ao final do evento. Foram coletados 93 formulários
preenchidos que resultaram nas seguintes estatísticas. Em 58\% das
respostas, os participantes se identificaram como desenvolvedores e
outros 25\% se disseram gerentes. 84\% tinham menos de um ano de
experiência com métodos ágeis e tinham menos de 35 anos de
idade. Estes resultados caracterizam uma população de jovens
profissionais com interesse em métodos ágeis mas sem conhecimentos
reais. Do ponto de vista do software livre, 68\% das respostas diziam
nunca ter contribuído com software livre e 25\% afirmavam colaborarem
ocasionalmente com algum projeto.

A partir desses resultados, conhecimento em métodos ágeis e
contribuições com software livre obtiveram uma correlação de 0,15
sendo que quanto mais novo, maior a probabilidade de que o indivíduo
tenha conhecimento em métodos ágeis (correlação de 0,25) e de que
contribua com software livre (correlação de 0,15). Essas correlações
são relativamente baixas mas nos dão uma idéia de que as comunidades
não estão muito relacionadas e que, portanto, é interessante pensar na
união desses duas comunidades como uma forma de diminuir os problemas
de desenvolvimento de software que é o objetivo desse trabalho.

%% ------------------------------------------------------------------------- %%
\section{Contribuições}
\label{sec:contribucoes}

As principais contribuições deste trabalho estão discriminadas abaixo:

\begin{itemize}
\item Uma primeira pesquisa relacionando métodos ágeis e software livre;
\item Uma pesquisa com a comunidade de software livre sobre sua relação com métodos ágeis;
\item Uma pesquisa com a comunidade de métodos ágeis sobre sua relação com software livre
\end{itemize}

%% ------------------------------------------------------------------------- %%
\section{Organização do Trabalho}
\label{sec:organizacao_trabalho}

Os tópicos apresentados nesse trabalho consideram apenas um
subconjunto de projetos que são ditos ágeis ou software livre. O
Capítulo \ref{cap:definicoes} apresenta as definições usadas nesse
trabalho. O Capítulo \ref{cap:foss} apresenta alguns aspectos da
maioria das comunidades de software livre que poderiam ser melhorados
com práticas ou princípios ágeis. O capítulo seguinte (Capítulo
\ref{cap:agile}) aborda alguns problemas que surgem na utilização de
métodos ágeis com equipes grandes e distribuídas que já foram
endereçados nas comunidades de software livre. Com essas
considerações, o Capítulo \ref{cap:perspectivas} apresenta os próximos
passos e as metas estabelecidas. Por fim, o Capítulo
\ref{cap:conclusoes} resume o trabalho realizado e apresenta os planos
para o futuro.
